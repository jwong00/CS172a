%        File: hw2.tex
%     Created: Tue Feb 10 04:00 PM 2015 P
% Last Change: Tue Feb 10 04:00 PM 2015 P
%
\documentclass[a4paper]{article}
\author{Jeremy Wong}
\title{CS-172a, Homework Assignment \#2}
\begin{document}
\maketitle
\section{Specification}
This is a bash script for grading a directory of multiple-choice homework assignments. The directory should contain a ``key'' file that lists the solutions, line-by-line.

The working directory should contain the script and the directory of assignments to grade. The directory of assignments should contain a file called ``key'' and any number of files containing submitted homework assignments.

When this script is run, it generates a subdirectory ``graded'' in the directory of assignments to grade. This contains graded versions of the assignment.
\section{Implementation}
\begin{verbatim}
#!/bin/bash
p=$PWD
dir=$1
echo "dir" is $dir

acount=0
for i in `cat $dir/key`
do
   acount=$((acount+1))
   A[$acount]=$i
   echo A'['$acount']' is $i
done

for i in `ls $dir`
do
   if [ "$i" != "key" ]
   then
      bcount=0
      gradedOut=$dir/graded/$i-Graded
      mkdir $dir/graded &>/dev/null
      touch $gradedOut 
      #echo $i
      #cat $dir/$i
      score=0
      for k in `cat $dir/$i`
      do
	 bcount=$((bcount+1))
	 if [ "$k" == ${A[$bcount]} ]
	 then 
	    echo $k CORRECT  >>$gradedOut
	    score=$((score+1))
	 else
	    echo $k WRONG >>$gradedOut
	 fi
      done
      echo Final grade: $score/$acount >>$gradedOut
   fi
done
\end{verbatim}

\section{Why this implementation is better}
\begin{enumerate}
   \item It's terser. Some unecessary lines were cut, like ``pushd students\ldots''
   \item It's unclear how the script was meant to be used. 
   \item Doesn't output a results.csv. This might be seen as not-a-feature, or worse than the original implementation, but this makes it simpler, and more conformant with the Unix philosophy. This script \textit{just} grades the assignments. Another script could be written to handle the generation of *.csv files. This would make the set of programs follow the rule of composition.

\end{enumerate}

\begin{verbatim}
#!/bin/bash
p=$PWD
>$p/results.hw$1 #output results to a new file in the current working dir

numq=`wc -l hw$1/answers |cut -f1 -d" "` #store number of questions

#store each line in array
acount=0
for i in `cat hw$1/answers` 
do
acount=$((acount+1))
A[$acount]=$i
# echo A'[' $acount ']' is $i
done

for i in `ls students`
do
	bcount=0
	pushd students/$i &>/dev/null
	#pushing a non-directory onto the directory stack seems bad
	echo now in $PWD &>/dev/null
	if [ -f hw$1 ] 
		#check for existence of file
		#check to see if file is regular
	then
		WRONG=""
		correct=0
		for j in `cat hw$1` #for each file in directory hw$1
		#this line doesnt execute properly
		do
			bcount=$((bcount+1))
			if [ "$j" != ${A[$bcount]} ]
			then
				WRONG=`echo $WRONG $j " "`
			else
				correct=$((correct+1))
			fi
		done
	s=$(echo "scale=2; $correct / $numq" |bc )
	echo $i SCORE $s wrong were $WRONG >>$p/results.hw$1
	else s=0
	fi
	printf "%s,%4.2f\n" $i $s >>$p/scores_hw$1.csv
	
	popd &>/dev/null
done
\end{verbatim}


\end{document}


